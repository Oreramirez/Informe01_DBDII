\section{MARCO TEORICO} 
\begin{itemize}
\subsection{Postman:}
	\item Gestiona y construye tus APIs rápidamente,Postman surgió originariamente como una extensión para el navegador Google Chrome. A día de hoy dispone de aplicaciones nativas para MAC y Windows y están trabajando en una aplicación nativa para Linux (disponible en versión beta).
	\item Está compuesto por diferentes herramientas y utilidades gratuitas (en la versión free) que permiten realizar tareas diferentes dentro del mundo API REST: creación de peticiones a APIs internas o de terceros, elaboración de tests para validar el comportamiento de APIs, posibilidad de crear entornos de trabajo diferentes (con variables globales y locales), y todo ello con la posibilidad de ser compartido con otros compañeros del equipo de manera gratuita (exportación de toda esta información mediante URL en formato JSON).
	\item Además, dispone de un modo cloud colaborativo (de pago) para que equipos de trabajo puedan desarrollar entre todos colecciones para APIs sincronizadas en la nube para una integración más inmediata y sincronizada.

\subsection{JSON :}
	\item Es un formato de texto sencillo para el intercambio de datos. 
          \item  Está constituído por dos estructuras:
Una colección de pares de nombre/valor. En varios lenguajes esto es conocido como un objeto, registro, estructura, diccionario, tabla hash, lista de claves o un arreglo asociativo.
Una lista ordenada de valores. En la mayoría de los lenguajes, esto se implementa como arreglos, vectores, listas o sequencias.
\subsection{API REST :}
          \item El término REST (Representational State Transfer) se originó en el año 2000, descrito en la tesis de Roy Fielding, padre de la especificación HTTP. Un servicio REST no es una arquitectura software, sino un conjunto de restricciones con las que podemos crear un estilo de arquitectura software, la cual podremos usar para crear aplicaciones web respetando HTTP. 

\end{itemize}
\subsection{CARACTERISTICAS DE API REST:}
 Las operaciones más importantes que nos permitirán manipular los recursos son cuatro.
 GET para consultar y leer, POST para crear, PUT para editar y DELETE para eliminar.

 El uso de hipermedios (término que en el ámbito de las páginas web define el conjunto de procedimientos para crear contenidos que contengan texto, imagen, vídeo, audio y otros métodos de información) para permitir al usuario navegar por los distintos recursos de una API REST a través de enlaces HTML.

